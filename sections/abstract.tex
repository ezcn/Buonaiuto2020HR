\section*{Abstract}

%All original research articles published in Human Reproduction are now required to have an extended abstract. The aim behind the change to this new format is to capture the essence, novelty and importance of each study, making the information more instantly available to readers. The abstract should clearly set out the research question, study design, findings, implications, funding and competing interests.

%Use MESH* terms in title and abstract.
%* for MESH terms see PubMed at http://www.ncbi.nlm.nih.gov/pubmed/
%Version 2.6     25/01/2013

%Please complete all sections; please do not remove or change headings.

%\subsection*{\textsc{Title}} add title and uncomment if abstract is sent separately 
%[if randomised, identify the study as such in title]

\subsection*{\textsc{Study question:}} 
%[A SINGLE question (ending in a question mark), limited to the PRIMARY objective of the study ONLY (do not include secondary questions)]

\subsection*{\textsc{Summary answer:}}  
%[The main conclusion. A single sentence, this should be limited to the primary results of the study, without any discussion of their implications]

\subsection*{\textsc{What is known already:}} 
%[One or two short sentences] 

\subsection*{\textsc{Study design, size, duration:}} 
%[Cross sectional – control versus treatment, longitudinal –time-course, age-course. Numbers of treated/controls, treatment duration, sampling procedures]

\subsection*{\textsc{Participants/materials, setting, methods:}} %[General approach used eg cell/tissue culture/transfection, animal treatments/models, transgenesis. Species, ages, gender, cell type. Methods and endpoints used – eg hormone, cytokine, growth factor measurements, cell numbers/proliferation, tissue morphology/composition, FACS, immunohistochemistry, Westerns, quantitative PCR, FISH]

\subsection*{\textsc{Main results and the role of chance:}} 
%[P values, biological gradient, repeatability/robustness, mechanisms identified/involved]

\subsection*{\textsc{Limitations, reasons for caution:}} 
%[Descriptive, only in vitro, cell transfection, shown only in one species, technical limitations and reasons for caution, cell/animal lethality in a knock-out, disease- or cell-specificity]

\subsection*{\textsc{Wider implications of the findings :}} 
%[Agreement/disagreement with literature, resolution of previous disparity, new insights/mechanisms in disease(s), new therapeutic potential, cell-, species- gender-, or age-implications, relevance to other systems]

\subsection*{\textsc{Study funding/competing interest(s):}} %Trial registration number: a trial registration number is only required for clinical trials.

