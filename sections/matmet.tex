\section*{Materials and methods}

%The names and country of origin of all suppliers should be included.
%Please use subheadings.
%The study population and participants should be described.
%A separate subheading within the materials and methods should describe the statistical analyses .
%A further separate subheading should detail any ethical approval for the use of animals or for the collection and use of human tissue.
\subsection*{Embryo data and samples collection}
The study protocol was examined by the Comitato Etico di Area Vasta Emilia Centro (CE-AVEC) of the Azienda Ospedaliero - Universitaria di Bologna Policlinico S. Orsola-Malpighi. The committee gave the ethical approval of the study (reference CE/FE 170475). All participants provided written informed consents before entering the study.
Cases were recruited at the Unit of Obstetrics and Gynecology of the Sant’Anna University Hospital in Ferrara, Italy, from 2017 to 2018.
The inclusion criteria were: age between 18 and 42 years and gestational age up to 12 weeks. Exclusion criterion was any clinical condition that could prevent full-term pregnancies. Known causes of pregnancy losses were excluded by standard diagnostic protocol including hysteroscopy, laparoscopy, ultrasound, karyotype analysis, detection of immunological risk factors (anticardiolipin, lupus anticoagulant, antinuclear antibodies) and hormonal status (gonadotrophins, FSH, LH, prolactin, thyroid hormones, thyroperoxidase) before inclusion in the study. Gestational weeks were calculated from the last menstrual period. Demographic, antropometric and clinical data of cases, including obstetric history, family history of malformations, and periconceptional supplementation with folic acid, were anonymized and linked to biological samples by coding. 


\subsection*{DNA preparation and sequencing} 
Retained product of conception was removed from uterus using a suction curette, and chorionic villi (CV) were carefully dissected from decidual tissue. We used dry homogenization after exploring a range of possibilities (Figure \ref{fig:dna}A). 
Genomic DNA was extracted from CV samples using QIAamp DNA Mini Kit (Ref: 51304, Qiagen) according to manufacturer’s protocol. This kit was chosen after considering the yield of two types of resin and one membrane (Figure \ref{fig:dna}B). DNA was titrated using Qubit 2.0 Fluorometer (Life Technologies).
Whole-genome sequencing of the genomic DNA extracted from chorionic villi was done through a service provider (Macrogen). In particular, libraries for sequencing were prepared using the Illumina TruSeq DNA PCR-free Library (insert size 350bp) and samples were sequenced at 30X mapped (110Gb) 150bp PE on HiSeqX. 

%Samples collection was done by the Unit of Obstetrics and Gynecology of the Sant’Anna University Hospital in Ferrara, Italy, from 2016 to 2020. It was approved by the local Ethical Committee (approval number CE/FE 170475) and carried out in compliance with the Helsinki Declaration. All participants provided written informed consents before recruitment. The inclusion criteria were: age in the range 18–42 years; gestational age within the first 12 weeks. Maximum gestational age for cases of voluntary termination of pregnancy was ninety days, according to the Italian law, namely Bill 194, Article 4. 

%Anonymous data about age, body mass index, menarche age, previous pregnancies, and geographical origin were considered for this study. Data cleaning, refining, and analysis (summary statistics, hypothesis testing) were performed using R \cite{R}.

%Fresh embryonic tissues were analyzed the same day of collection. Chorionic villi (CV) were separated from the maternal decidua in sterility under a hood using a stereomicroscope (Leica Microsystems Srl, All Microscopy and Histology, Milan, I-20142 Italy). The villi were stored at -20\textdegree  C for a few months or at 4\textdegree  C in RPMI media for not more than a week before proceeding to homogenization and DNA extraction. 

%We explored a range of possibilities for DNA preparation from CV that includes two methods of tissue homogenization and three methods of DNA isolation. We do not observe significant difference between homogenization techniques, therefore we proceeded with dry homogenization that is technically less challenging (Figure \ref{fig:dna}A). Similarly, in the case of DNA isolation we considered two types of resin and one membrane and we do not observe significant differences in yield neither among the techniques or among samples from maternal blood, voluntary termination of pregnancy and miscarriages but slightly higher range of yield for one type of resin (Figure \ref{fig:dna}B and Figure \ref{fig:dnayeld}). Quantification of genomic DNA was done with Qubit® 2.0 Fluorometer (Invitrogen) according to manufacture instructions.

%Genomic DNA (gDNA) was extracted from chorionic villi dissected from abortion tissue specimens using QIAamp DNA Blood Mini Kit (Qiagen), and with Instagene TM Matrix (Bio-Rad) according to manufacturer protocols \textit{(QIAamp DNA Mini and Blood Mini Handbook 05/2016. Instruction Manual, InstaGene™ Matrix, LIT544 Rev G)}



\subsection*{Detection of chromosomal aneuploidies in embryos} 
%\subsubsection*{Detection of sex and numerical anomalies through quantitative PCR}
A rapid screening of sex and anuploidies for chromosomes 13, 15, 16, 18, 21, 22 and X was carried out on geomic DNA extracted from the chorionic villi performing five multiplex Quantitative Fluorescent PCR (QFPCR) assays. QFPCR assays were performed in a total volume of 25μl containing 40–100ng of genomic DNA, 10mM dNTP (Roche), 6-30 pmol final concentration of each primer, 1×Fast taq polymerase buffer (15mmol/l MgCl 2 ) (Roche), and 2.5 U of Fasta taq polymerase (Roche). QFPCR conditions were as follows: denaturation at 95\textdegree C for 10 min followed by 10 cycles consisting of melting at 95\textdegree C for 1 min, annealing at 65\textdegree C (-1\textdegree C / cycle) for 1 minutes, and then extension at 72\textdegree C for 40 seconds, then for 23 cycles at 95\textdegree C for 1 min, 55\textdegree C for 1 min, and 72\textdegree C for 1 min. Final extension was for 10min at 72\textdegree C and at 60\textdegree C for 60 min. Fluorescence-labelled QFPCR products were electrophoresed in an CEQ 8000 Backman by combining 40 μl of Hi-Di Formamide and 0.5 μl of DNA size standard 400 (Backman); QFPCR products were visualized and quantified as peak areas of each respective repeat lengths. In normal heterozygous subjects, the QFPCR product of each STR should show two peaks with similar fluorescent activities and thus a ratio of peak areas close to 1:1 (ranging from 0.8 to 1.4:1). A trisomy is suspected when the ratio is  above or below this range (peak area ratios ≤ 0.6 and ≥ 1.8, trisomic diallelic pattern), otherwise there are three alleles of equal peak area with a ratio of 1:1:1 (trisomic triallelic). The presence of trisomic triallelic or diallelic patterns for at least two different STRs on the same chromosome is considered as evidence of trisomy. Trisomic patterns observed for all chromosome-specific STRs are indicative of triploidy. Therefore accurate X chromosome dosage, to perform diagnosis of X monosomy, can be assessed by TAF9L marker allowing This gene has a high degree of sequence identity between chromosome 3 and chromosome X; primers on this gene amplify a 3 b.p. deletion generating a chromosome X specific product of 141 b.p. and a chromosome 3 specific product of 144 b.p. Maternal contamination was also checked by QFPCR comparing the alleles found in miscarriages with those found in maternal blood. %Samples with $>$20\% contamination were not included in the study.


%\subsubsection*{Comparative Genomic Hybridization} 
%-- MERIGEN Agilent SurePrint G3 CGH-only. Log-ratio produced by the Agilent CytoGenomics v3.0.4 software 

Comparative Genomic Hybridization was carried out using the Agilent SurePrint G3 Human CGH Microarray. Samples underwent DNA quantification and quality analysis prior to be labeled and hybridized on the microarray. Following hybridization samples were washed and the chip was scanned at 3 microns using the Agilent SureScan Microarray Scanner. The LogRatio from the arrays were segmented into regions of estimated equal copy number using both the method implemented in theAgilent CytoGenomics V3.0.4 software, and the Penalized least square implemented in the R package Copynumber (PLS, \cite{nilsen2012copynumber}). Classification as copy number of gains or losses (copy number variants) was done using as criteria at least five probes and Zscore $<$0.0016 (SD*4)\cite{vermeesch2005molecular}. 
 %Circular Binary Segmentation (CBS)\cite{Venkatraman2018}, 
 
 %First using the Feature Extraction (part of Agilent CytoGenomics or Agilent Genomic Workbench) then seamlessly translates the resulting image into log ratios and uses advanced statistical analysis to associate these with correlative QC metrics. 
%Results can be further analyzed and displayed for biological interpretation in Agilent CytoGenomics V3.0.4 software.
%Microarray-based comparative genomic hybridization (array CGH) is a powerful method for the genome wide detection of chromosome copy number changes at a higher resolution level than conventional chromosome-based CGH.
%The Agilent SurePrint G3 CGH Microarrays kit was used for this study.
%following a six steps protocol is composed by 6 steps:
%Step 1. DNA Quantitation and Quality Analysis 
%Step 2. Sample Preparation 
%Step 3. Sample Labeling 
%Step 4. Microarray Hybridization 
%Step 5. Microarray Washing 
%Step 6. Microarray Scanning and Analysis 
%SurePrint G3 CGH Microarrays are scanned at 3 microns using the Agilent SureScan Microarray Scanner. Feature Extraction (part of Agilent CytoGenomics or Agilent Genomic Workbench) then seamlessly translates the resulting image into log ratios and uses advanced statistical analysis to associate these with correlative QC metrics. 
%Results can be further analyzed and displayed for biological interpretation in Agilent CytoGenomics V3.0.4 software.




%\subsection*{DNA sequencing}

%SILVIA We obtained XX GB of data per individual, with.95.8 of the genome covered more than XX times.

\subsection*{Statistical and sequence analyses}
Data cleaning, refining, and analysis (summary statistics, hypothesis testing) were performed using R \cite{R}.
Reads in the FASTQ file sequence data were aligned against the reference genome GRChg38.p12 using \textsc{bwa}\cite{li2013aligning} and \textsc{samtools}\cite{li2011statistical}. Variant calling was done using \textsc{freebayes}\cite{garrison2012haplotype}. The resulting VCF files were refined in further steps: \textsc{vcffilter} \cite{vcflib} was used to filter variants for quality score$>$20, leaving only variants with estimated 99\% probability of a polymorphic genotype call; \textsc{vt}\cite{tan2015unified} was used to normalize variants and deconstruct multiallelic variants. Refined VCF files were compressed and indexed using samtools \cite{li2011statistical}. Variants were annotated for functional effects and allele frequency in other populations using Variant Effect Predictor\cite{mclaren2016ensembl}. Phasing was done using Beagle 5.1\cite{browning2018one} under standard parameters.  

Principal component analysis was done with PLINK\cite{chang2015second} using 1,2M autosomal SNPs. 

The \gp pipeline for variant prioritization is written in Python and R and the code is publicly available (\url{https://github.com/ezcn/grep}). The manually curated list of genes associated with miscarriages (recurrent and spontaneous) was obtained through a comprehensive search of the published literature. We considered seven studies highlighting the association of genes with miscarriages \cite{colley2019potential, fu2018whole, laisk2020genetic, pereza2017systematic, qiao2016whole, quintero2017novel, rull2012genetics}. This compendium was further supplemented by genes from curated repositories such as Human Phenotype Ontology (HPO) [Robinson et al., 2008; URL: https://hpo.jax.org/app/browse/term/HP:0200067 last accessed: 1/12/2020 11:01:00 PM] and DisGeNET [Piñero et al., 2015; URL: http://www.disgenet.org/search last accessed: 1/12/2020 11:12:00 PM]. The search terms used were “recurrent miscarriages”, “abortion”, “spontaneous abortion”, and “recurrent spontaneous abortion”. After filtering by removing the duplicates, combining the gene sets obtained from the literature and databases yielded a total of 608 unique genes (Supplementary Table 1). Additional information of genes such as HGNC symbol, HGNC ID, Gene Stable ID, Chromosomal coordinates (GRChg38), karyotype band, transcript count, protein stable ID were extracted from Ensembl Biomart\cite{kinsella2011ensembl}. %[Kinsella et al., 2011, Yates et al., 2020]. %Furthermore, Haploinsufficiency score (HI index) and Loss Intolerance (pLI) values of all the genes were compiled from DECIPHER [Firth et al., 2009] and gnomAD [Karczewski et al., 2020] respectively. 

Overrepresentation tests and protein classification were performed using the R package ReactomePA\cite{yu2016reactomepa}.  






